%----------------------------------------------------------------
%
%  File    :  chapter1.tex
%
%  Authors : Michael Fuska, FH Campus Wien, Austria% 
%  Created : 13 Feb 2016
%
%  Changed :  
% 
%----------------------------------------------------------------


\chapter{Einleitung}
\label{ch:intro}


\section{Einführung}
\label{sec:Einführung}

Das Smartphone ist ein ständiger Begleiter des Menschen geworden. Es nimmt von Tag zu Tag mehr Einfluss auf unser Leben. Neben dem privaten Gebrauch ist es auch im beruflichen Alltag nicht mehr wegzudenken. Sensible Informationen und Daten werden auf dem Smartphone verarbeitet und/oder gespeichert. Daraus ergibt sich, dass die Sicherheit und Stabilität der Smartphones zunehmend an Bedeutung gewinnen muss.\par 
Am 29.Juni 2007 wurden das erste iPhone mit der iOS Version 1.0 von Apple zum Verkauf angeboten. Elf Tage nach Verkaufsstart war der erste Jailbreak für diese iOS Version verfügbar. Das Ziel eines Jailbreak ist es, \textit{\glqq Root-Zugriffsrechte\grqq{}} für das Betriebssystem zu erhalten. Der \textit{\glqq Root-User\grqq{}} hat vollen Zugriff auf das Betriebssystem und kann mit diesen Rechten unter anderem, unautorisierte Software auf dem iOS Device installieren.\par 

Es gibt für \textit{\glqq gejailbreakte\grqq{}} iOS Devices, neben dem offiziellen App Store (Apple), zwei weitere App Stores
\begin{itemize}
    \item Cydia
    \item Taig
\end{itemize}
Diese beiden App Stores unterliegen nicht der Sicherheitsüberprüfung von Apple. Der User muss sich über die möglichen Konsequenzen bewusst sein, wenn er eine Applikation aus einem nicht offiziellen App Store installiert. Neben Malware, die direkt in der heruntergeladen und installierten App eingebettet sein kann, ermöglichen Programme wie 
\begin{itemize}
    \item SSH Dämon,
    \item Terminalprogramm,
    \item und Shells (sh, bash)
\end{itemize}
weitere Angriffsvektoren für \textit{\glqq Hacker\grqq{}}.

Ein Jailbreak nutzt \textit{\glqq Bugs\grqq{}} im iOS aus und umgeht so die Sicherheitsmaßnahmen von Apple. Für Apple bedeutet jeder Jailbreak ein Aufzeigen von Sicherheitsschwächen des Apple Betriebssystems. Durch den öffentlichen Druck und die Möglichkeit, dass diese Sicherheitsmängel für Angriffe ausgenutzt werden könnten, wird Apple immer wieder gezwungen, Zeit und Geld in die Verbesserung der iOS Sicherheit zu investieren. 

Am 21.September 2015 wurde von der Firma \textit{\glqq Zerodium\grqq{}} ein Preisgeld von einer Million U.S Dollar für einen \textit{\glqq zero day bug\grqq{}} für die iOS Version 9.x ausgeschrieben. Das Ziel dieser Ausschreibung war es, einen System Remote-Zugriff zu erhalten. 
\begin{description}
    \item[\parbox{\textwidth} {Folgende iOS Sicherheitsmechanismen mussten durch diesen Jailbreak umgangen werden}]~\par
    \begin{enumerate}
        \item ASLR, 
        \item Sandbox,
        \item Root Zugriff, 
        \item Code Signing, 
        \item und der Secure Boot Chain.
    \end{enumerate}
\end{description} 

Dieser unerlaubte Zugriff auf das Smartphone musste vor dem User verborgen bleiben. Diese Ausschreibung zeigt, dass nur mehr sehr wenige Menschen die Fähigkeiten besitzen, Sicherheitslücken des mobilen Betriebssystems von Apple auszunützen. Am ersten November 2015 endete die Ausschreibung und es wurde bekanntgegeben, dass das Preisgeld einmal ausbezahlt wurde. 

Am 29.September 2015 nahm Apple Stellung  zu den \textit{\glqq backdoor\grqq{}} Gerüchten in der Presse. Laut Aussage von Apple gibt es ab der iOS Version 8 keine Möglichkeit mehr, an die \textit{\glqq Cleartext Userdaten\grqq{}} eines Devices zu gelangen. Apple begründet dies in ihrer Aussendung so:
\begin{enumerate}
    \item \textbf{Es gibt für Apple keine Möglichkeit, den Passcode des iOS Devices zu umgehen.}
    \item \textbf{Die Verschlüsselungskeys werden von Apple nicht mehr gespeichert.}
\end{enumerate}

Diese Ausgabe wird durch einen weiteren Fall untermauert. Das FBI forderte im Dezember 2016 Apple auf, dem FBI zu helfen und eine iOS Device zu entsperren. Das FBI benötigte einen Zugang zu den Userdaten des Gerätes. Selbst die in diesem Fall vertrauten Gerichte, konnten Apple nicht verpflichten, das iPhone zu entsperren. Zur Zeit arbeitet der US Kongress an einem neuem Gesetzt, in dem festgelegt werden soll, wie mit solchen Fällen umgegangen werden soll. Ein Teil dieser Arbeit wird es sein, festzulegen, ob Firmen wie Apple gezwungen werden können, \textit{\glqq Backdoors\grqq{}} in ihrer Software zu implementieren. Dies würde es staatlichen Organisation erleichtern, Zugriff auf die Daten von Usern erhalten, dies hat den Beigeschmack, dass dieses Backdoor auch von jedem anderen benutzt werden kann, sobald dieser, das Backdoor entdeckt würde.

\section{Motivation }
\label{sec:IntroMotivation}
Das iPhone ist nun seit über sechs Jahren mein ständiger Begleiter, sowohl im privaten als auch im beruflichen Umfeld. Die unzureichende Möglichkeit das iPhone meinen Bedürfnissen anzupassen, war ausschlaggebend für mich, mich mit dem Thema iOS Jailbreak auseinander zusetzten. 
Für mich ist weiters wichtig, dass nach der Installation eines Jailbreaks die klassischen Ziele der Informationssicherheit
\begin{enumerate}
    \item die \textbf{Integrität,}
    \item die \textbf{Vertraulichkeit} 
    \item und die \textbf{Verfügbarkeit der Daten,}
\end{enumerate}
sowie die \textbf{Stabilität des System} erhalten bleiben.


\section{Aufbau dieser Arbeit}
\label{sec:IntroAufbau}
In dieser Sektion wird der Aufbau der Arbeit beschrieben. \textbf{Das Kapitel \ref{ch:intro}} befasst sich mit einer allgemeinen Beschreibung des Themas, unteranderem gibt es einen Überblick bzw. Einblick in die weiteren Kapiteln dieser Arbeit.

%\paragraph{Im Kapitel \ref{ch:DF}} werden die grundlegenden Begriffe und Methoden der digitalen Forensik beschreiben. Es werden vor allem die Möglichkeiten angeführt, wie die Cleartext-Daten eines iOS Device extrahierte werden können.

\paragraph{Im Kapitel \ref{ch:JB}} werden die unterschiedlichen Arten von Jailbreaks beschrieben, sowie statistische Aufbereitung zu diesem Thema.

\paragraph{Im Kapitel \ref{ch:iOS}} wird das mobile Betriebssystem von Apple im Detail beschrieben. Es werden die Betriebssystemanforderungen auf den unterschiedlichen Layer des mobilen Betriebssystem von Apple eingegangen. Die iOS Architektur wird in diesem Kapitel im Detail erklärt. 

\paragraph{Im Kapitel \ref{ch:iOSSicherheitsKonzepte}} werden die einzelnen iOS Sicherheitsmechanismen beschrieben die von Apple im Laufe Zeit implementiert worden sind.

\paragraph{Im Kapitel \ref{ch:Ergebnisse}}

\paragraph{Im Kapitel \ref{ch:Conclusion}}

\subsection{Problem Defintion}
\label{sec:IntroProblem}
User verwenden JB, keine Wissen über die Folgen
Ein JB führt umfangreiche Änderungen in iOS durch diese bleiben vom User verborgen.
Die Stabilität des Systems ist für den User merkbar -> wenn unzufrieden installation der originalen SW
\begin{description}
    \item[\parbox{\textwidth} {Aber Schutzziele Daten erhalten? }]~\par
    \begin{itemize}
       \item Vertraulichkeit (confidentiality): Daten dürfen lediglich von autorisierten Benutzern gelesen bzw. modifiziert werden, dies gilt sowohl beim Zugriff auf gespeicherte Daten, wie auch während der Datenübertragung.
       \item Integrität (integrity): Daten dürfen nicht unbemerkt vom User verändert werden.
      \item  Verfügbarkeit (availability): Verhinderung von Systemausfällen; der Zugriff auf Daten muss innerhalb eines vereinbarten Zeitrahmens gewährleistet sein.
       \item   Authentizität (authenticity) bezeichnet die Eigenschaften der Echtheit, Überprüfbarkeit und Vertrauenswürdigkeit der Daten.
       \item Nichtabstreitbarkeit (non repudiation): Erfordert das Handlungen eindeutig zugeordnet werden können und diese nicht abgestritten werden könnne
       \item  Zurechenbarkeit (accountability): Eine durchgeführte Handlung kann einem Kommunikationspartner eindeutig zugeordnet werden.
       \item Anonymität
    \end{itemize}
\end{description} 

               

\subsection{Ziel dieser Arbeit}
\label{sec:IntroZiel}
Eine Aussage über die Sicherheit und die Vertrauenswürdigkeit des iDevices zu tätigen 
und welchen Einfluss die JB darauf hatten.
    \begin{description}
        \item[\parbox{\textwidth} {Daraus ergeben sich folgende Fragestellungen}]~\par
            \begin{enumerate}
                \item \label{frage:1} \textbf{Welche Faktoren sind für die Sicherheit und die Vertrauenswürdigkeit eines iOS Device ausschlaggebend (Kapitel: \ref{sec:Frage1})? } 
                    
            \item \label{frage:2}\textbf{Welche Auswirkung haben die von Apple eingeführten Sicherheitsmechanismen und Sicherheitsupdates auf die Sicherheit des Systems (Kapitel: \ref{sec:Frage2})?} 
            
            \item \label{frage:3}\textbf{Welche Auswirkung hat die Veröffentlichung eines JB auf die Sicherheit des Devices(Kapitel: \ref{sec:Frage3})?}    
     
    \end{enumerate}
\end{description} 
 
 
\subsection{Methodik dieser Arbeit}
\label{sec:MethArbeit}

    \begin{enumerate}
        \item Sammeln der Daten
        \begin{itemize}
            \item Literaturrecherche
            \begin{itemize}
                \item iOS
                \item HW
                \item JB
            \end{itemize}
            \item Historische Daten
            \item Aktuelle Daten 
        \end{itemize}
        \item Aufbereitung der Daten
        \item Analyse 
    \end{enumerate}


\subsection{Rahmenbedingungen}
\label{sec:Rahmenbedingungen}
Datenerfassung gelten folgende Rahmenbedingungen:
\begin{itemize}
    \item ab iOS Version 3.0 bis 9.3.2
    \item iOS HW inklusive 6s
    \item Technischer Stand per 17.07.2016
    \item JB bis 17.07.2016
\end{itemize}






    