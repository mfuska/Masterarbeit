%----------------------------------------------------------------
%
%  File    :  chapter1.tex
%
%  Authors : Michael Fuska, FH Campus Wien, Austria% 
%  Created : 13 Feb 2016
%
%  Changed :  
% 
%----------------------------------------------------------------


\chapter{Einleitung}
\label{ch:intro}


\section{Einführung}
\label{sec:Einführung}

Das Smartphone ist ein ständiger Begleiter des Menschen geworden. Es nimmt von Tag zu Tag mehr Einfluss auf unser Leben. Neben dem privaten Gebrauch ist es auch im beruflichen Alltag nicht mehr wegzudenken. Sensible Informationen und Daten werden auf dem Smartphone verarbeitet und/oder gespeichert. Daraus ergibt sich, dass die Sicherheit, Vertrauenswürdigkeit und Stabilität der Smartphones zunehmend an Bedeutung gewinnen müssen.\par 
Am 29 Juni 2007 wurde das erste iPhone mit der iOS Version 1.0 von Apple zum Verkauf angeboten. Elf Tage nach Verkaufsstart war der erste Jailbreak (JB) für diese iOS Version verfügbar. Das Ziel eines Jailbreak ist es, \textit{\glqq root-Zugriffsrechte\grqq{}} für das Betriebssystem zu erhalten. Der \textit{\glqq root-User\grqq{}} hat vollen Zugriff auf das Betriebssystem und kann mit diesen Rechten unter anderem unautorisierte Software auf dem iOS Device installieren.\par 
Es gibt für \textit{\glqq gejailbreakte\grqq{}} iOS Devices, neben dem offiziellen Apple App Store (iTunes), zwei weitere App Stores
    \begin{itemize}
      	\item \textbf{Cydia} \footnote{\url{https://cydia.saurik.com/}}
		\item \textbf{TaiG} \footnote{\url{http://www.taig.com/en}}
    \end{itemize}
Diese beiden App Stores unterliegen nicht der Sicherheitsüberprüfung von Apple. Der User muss sich über die möglichen Konsequenzen bewusst sein, wenn er eine Applikation aus einem nicht offiziellen App Store installiert. Neben Malware, die direkt in der heruntergeladenen und installierten App eingebettet sein kann, ermöglichen Programme wie 
\begin{itemize}
    \item SSH Dämon,
    \item Terminalprogramm,
    \item und Shells (sh, bash)
\end{itemize}
weitere Angriffsvektoren für \textit{\glqq Hacker\grqq{}}.
Ein Jailbreak nutzt \textit{\glqq Bugs\grqq{}} im iOS aus und umgeht so die Sicherheitsmaßnahmen von Apple. Für Apple bedeutet jeder Jailbreak ein Aufzeigen von Sicherheitsschwächen des Apple Betriebssystems. Durch den öffentlichen Druck und die Möglichkeit, dass diese Sicherheitsmängel für Angriffe ausgenutzt werden könnten, wird Apple immer wieder gezwungen, Zeit und Geld in die Verbesserung der iOS Sicherheit zu investieren. 
Am 21 September 2015 wurde von der Firma \textit{\glqq Zerodium\grqq{}} ein Preisgeld von einer Million US Dollar für einen \textit{\glqq zero day bug\grqq{}} für die iOS Version 9.x ausgeschrieben. Das Ziel dieser Ausschreibung war es, einen System Remote-Zugriff zu erhalten. 
\begin{description}
    \item[\parbox{\textwidth} {Folgende iOS Sicherheitsmechanismen mussten durch diesen Jailbreak umgangen werden}]~\par
    \begin{enumerate}
        \item ASLR, 
        \item Sandbox,
        \item root-Zugriff, 
        \item Code Signing, 
        \item und der Secure Boot Chain.
    \end{enumerate}
\end{description} 
Dieser unerlaubte Zugriff auf das Smartphone musste vor dem User verborgen bleiben. Diese Ausschreibung zeigt, dass nur mehr sehr wenige Menschen die Fähigkeiten besitzen, Sicherheitslücken des mobilen Betriebssystems von Apple auszunützen. Am ersten November 2015 endete die Ausschreibung, und es wurde bekanntgegeben, dass das Preisgeld einmal ausbezahlt wurde. 
Am 29.September 2015 nahm Apple Stellung  zu den \textit{\glqq backdoor\grqq{}} Gerüchten in der Presse. Laut Aussage von Apple gibt es ab der iOS Version 8 keine Möglichkeit mehr, an die \textit{\glqq Cleartext Userdaten\grqq{}} eines Devices zu gelangen. Apple begründet dies in ihrer Aussendung so:
\begin{enumerate}
    \item \textbf{Es gibt für Apple keine Möglichkeit, den Passcode des iOS Devices zu umgehen.}
    \item \textbf{Die Verschlüsselungskeys werden von Apple nicht mehr gespeichert.}
\end{enumerate}
Diese Aussagen werden durch einen weiteren Fall untermauert. Das FBI forderte im Dezember 2015 Apple auf, dem FBI zu helfen und eine iOS Device zu entsperren. Das FBI benötigte einen Zugang zu den Userdaten des Gerätes. Selbst die in diesem Fall vertrauten Gerichte konnten Apple nicht verpflichten, das iPhone zu entsperren. Zurzeit arbeitet der US Kongress an einem neuen Gesetz, in dem festgelegt werden soll, wie mit solchen Fällen umgegangen werden soll. Ein Teil dieser Arbeit wird es sein, festzulegen, ob Firmen wie Apple gezwungen werden können, \textit{\glqq Backdoors\grqq{}} in ihrer Software zu implementieren. Dies würde es staatlichen Organisationen erleichtern, Zugriff auf die Daten von Usern zu erhalten. Dies hat aber den Beigeschmack, dass dieses Backdoor auch von jedem anderen benutzt werden kann, sobald dieser das Backdoor entdeckten würde. 

\section{Motivation }
\label{sec:IntroMotivation}
Das iPhone ist nun seit über sechs Jahren mein ständiger Begleiter, sowohl im privaten als auch im beruflichen Umfeld. Die unzureichende Möglichkeit, das iPhone meinen Bedürfnissen anzupassen, war ausschlaggebend mich mit dem Thema iOS Jailbreak auseinanderzusetzen. 
Für mich ist weiters wichtig, dass nach der Installation eines Jailbreaks die klassischen Ziele der Informationssicherheit
\begin{enumerate}
    \item die \textbf{Integrität,}
    \item die \textbf{Vertraulichkeit} 
    \item und die \textbf{Verfügbarkeit der Daten,}
\end{enumerate}
sowie die \textbf{Stabilität des Systems} erhalten bleiben.


\section{Aufbau dieser Arbeit}
\label{sec:IntroAufbau}
In dieser Sektion wird der Aufbau der Arbeit beschrieben. \textbf{Das Kapitel \ref{ch:intro}} befasst sich mit einer allgemeinen Beschreibung des Themas, unter anderem gibt es einen Überblick bzw. Einblick in die weiteren Kapitel dieser Arbeit. Außerdem wird die empirische Methode erklärt, mit der die erhobenen Daten aufbereitet und analysiert werden. 

\paragraph{Im Kapitel \ref{ch:JB}} werden die unterschiedlichen Arten von Jailbreaks beschrieben, sowie die statistische Aufbereitung zu diesem Thema.

\paragraph{Im Kapitel \ref{ch:iOS}} wird das mobile Betriebssystem von Apple im Detail beschrieben. Es wird auf die Betriebssystem-Anforderungen auf den unterschiedlichen Layern des mobilen Betriebssystems von Apple eingegangen. Die iOS Architektur wird in diesem Kapitel im Detail erklärt. 

\paragraph{Im Kapitel \ref{ch:iOSSicherheitsKonzepte}} werden die einzelnen iOS Sicherheitsmechanismen beschrieben, die von Apple im Laufe der Zeit implementiert worden sind.

\paragraph{Im Kapitel \ref{ch:Ergebnisse}} werden die Daten anhand der im Kapitel \ref{ch:intro} definierten Fragestellungen aufbereitet. Die Zielfragestellungen die im Kapitel \ref{ch:intro} angeführt wurden, werden mit den erstellten Metriken zu den Fragestellungen beantwortet.

\paragraph{Im Kapitel \ref{ch:Conclusion}} werden die Ergebnisse zusammengefasst, und die daraus resultierenden Schlüsse dokumentiert.

\section{Methodik dieser Arbeit}
\label{sec:MethArbeit}
In dieser Arbeit wird die Goal Question Metric (GQM) Methode zur Analyse der erhobenen Daten herangezogen. Die GQM Methode wird in der Softwareentwicklung verwendet um spezifische Qualitätsmodelle systematisch erstellen zu können (vgl. \cite{GQM[1], GQM[2]}).  Hierbei ist vor allem die klar strukturierte Vorgehensweise dieses Systems hervorzuheben.\par

Briand beschreibt in seiner Arbeit die GQM Methode wie folgt: 
\begin{quote}
\textit{\glqq Goal-oriented measurement is the definition of a measurement program based on explicit and precisely defined goals that state how measurement will be used. In addition, explicit models have to be defined to support the derivation of measures from the goals in a traceable and unambiguous manner.\grqq{}} \cite{GQM[5]} 
\end{quote}

Basili begründet die Verwendung der GQM Methode für den Softwareentwicklungsprozess so: 
\begin{quote}
    \textit{\glqq Any engineering process requires feedback and evaluation. Software development is an engineering discipline and measurement is an ideal mechanism for feedback and evaluation. The measurement and information fed back to all parties, e.g., developers, managers, customers and the corporation, help in the understanding and control of the software process and products, and the relationships between them. It helps in making intelligent decisions and improving over time. But measurement must be focused, based upon goals and models. We need to establish goals for the various software process and products and models. We need to establish the goals for the various software process and products and these goals should be measurable, driven by the appropriate models.\grqq{}} \cite{GQM[4]} 
\end{quote}
 
Dadurch soll die Nachvollziehbar- und Beweisbarkeit dieser Arbeit gewährleistet werden (vgl. \cite{GQM[3], GQM[4]}).
\begin{description}
    \item[\parbox{\textwidth} {Die GQM-Methode ist in drei Schritte unterteilt und dient zur Bewertung der erhobenen Daten}]~\par
    \begin{enumerate}
        \item Die Definition des Ziels (Goals)
        \item Die Definition von quantifizierbaren Fragen, deren Beantwortung eine Aussage zulässt, ob das Ziel erreicht wurde (Questions)
        \item Die Definition der Metriken, die eine messbare Beantwortung der Fragen darstellen (Metrics)
    \end{enumerate}
\end{description} 

\newpage
\subsection{GQM - Goal Definition}
\label{sec:GQMGoal}

\paragraph{G1:} \textit{\glqq Das Ziel dieser Arbeit ist es, eine Aussage über die Sicherheit und die Vertrauenswürdigkeit der iOS Produkte tätigen zu können.\grqq{}} \par 
\textbf{G1.1:} \textit{\glqq Erstes abgeleitetes Ziel ist es, einen Zusammenhang zwischen der Sicherheit und der Vertrauenswürdigkeit des iOS Device und der verwendeten Hardware der iOS Geräte herzustellen.\grqq{}} \par 
\textbf{G1.2:} \textit{\glqq Zweites abgeleitetes Ziel dieser Arbeit ist es, einen Zusammenhang zwischen der Sicherheit und der Vertrauenswürdigkeit des iOS Device und der installierten iOS Software herzustellen.\grqq{}}

\subsection{GQM - Fragen Definition}
\label{sec:GQMFragen}

\paragraph{Q1:} \textit{\glqq Welche Faktoren sind für die Sicherheit und die Vertrauenswürdigkeit eines iOS Device ausschlaggebend (siehe Kap. \ref{sec:Frage1})?\grqq{}}
                    
 \paragraph{Q2:} \textit{\glqq Welche Auswirkung haben die von Apple eingeführten Sicherheitsupdates auf die Sicherheit des Systems (siehe Kap. \ref{sec:Frage2})?\grqq{}}
        
\subsection{Hypothese}
\label{sec:GQMHypothese}
\paragraph{H1:} \textit{\glqq Die Anzahl der Tage, die die JB-Community benötigt, um ein JB für eine iOS Version oder ein iOS Device zu veröffentlichen, kann als Indikator für die Sicherheit und Vertrauenswürdigkeit der iOS Produkte herangezogen werden.\grqq{}}

\subsection{GQM - Metrik Definition}
\label{sec:GQMMetrik}

Damit die Softwarequalität messbar gemacht werden kann, wurden unterschiedliche Modelle für diesen Prozess entwickelt. Im ISO Standard ISO/IEC 25010:2011 werden die unterschiedlichsten Softwaremerkmale definiert und festgehalten. \cite{IOS25010} Unter anderem werden die Merkmale für die Zuverlässigkeit einer Software in diesem Standard definiert. Ein Software-Sicherheitsmerkmal ist die Anzahl der Softwarebugs, die in einer Software gefunden worden sind. In dieser Arbeit werden die geschlossenen Bugs der Apple-Sicherheitsupdates für diese Metrik herangezogen.
\begin{description}
    \item[\parbox{\textwidth} {In dieser Arbeit werden zwei Arten von Bugs unterschieden, die in den iOS Sicherheitsupdates geschlossen werden}]~\par
    \begin{enumerate}
        \item Softwarebugs, die in einem JB verwendet wurden, um ein JB auf einer früheren iOS Version durchzuführen. 
        \item Softwarebugs, die von Apple und anderen Sicherheitsforschern entdeckt und gemeldet wurden.
    \end{enumerate}
\end{description} 
 \par
\textbf{Der zweite Ansatz} bezieht sich auf die Anzahl der Tage, die benötigt wurden, um für eine neue iOS Version ein JB zu veröffentlichen. Hierfür werden nur \textit{\glqq untethered Jailbreak Daten\grqq{}} verwendet, damit die Vergleichbarkeit der Daten in dieser Arbeit gewährleistet ist.\par 
Es werden in dieser Arbeit zwei Arten von Metriken für die Aufbereitung der Daten verwendet.
\begin{enumerate}
    \item \textit{\glqq External metrics\grqq{}}\\
    Diese Art der Metrik ermöglicht es, die laufende Software extern zu bewerten.
    \item \textit{\glqq Quality in use metrics\grqq{}} \\
    Diese Art der Metrik ermöglicht es, eine finale Softwareversion zu bewerten. Dadurch wird die Bewertung der finalen Softwarequalität ermöglicht.
\end{enumerate}


%\section{Rahmenbedingungen}
%\label{sec:Rahmenbedingungen}
%In diesen Kapitel werden die Rahmenbedingungen für diese Arbeit zusammengefasst. Die Daten wurden von  unterschiedlich Quellen 
%\begin{description}
%    \item[\parbox{\textwidth} {Es werden nur folgenden}]~\par
%    \begin{enumerate}
%        \item  iOS Produkte betrachtet: iPhone 1 bis iPhone SE (siehe Tabelle: \ref{tab:iOSHW})
%        \item iOS Version betrachtet: iOS Version 3.0 bis 9.3.2
%        \item JBs betrachtet. JBs die bis zum 01.07.2016 veröffentlicht wurden.
%    \end{enumerate}
%  \end{description} 
%








    