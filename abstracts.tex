%----------------------------------------------------------------
%
%  File    :  abstracts.tex
%
%  Authors :  Keith Andrews, IICM, TU Graz, Austria
%             Manuel Koschuch, FH Campus Wien, Austria
% 
%  Created :  22 Feb 96
% 
%  Changed :  30 Oct 2008
% 
%----------------------------------------------------------------


% --- German and English Abstracts ------------------------------------------------

% --- German Abstract ----------------------------------------------------
\cleardoublepage

\begin{center}
{\Large\bfseries Kurzfassung}
\end{center}
In dieser Arbeit werden die Sicherheit und Vertrauenswürdigkeit der iOS Produkte untersucht. Im Detail werden die Sicherheitsmechanismen der iOS Produkte beschrieben und die Zusammenhänge erklärt. Ein weiterer Teil dieser Arbeit setzt sich mit den Jailbreak von iOS Devices / Versionen auseinander. Es wird angeführt, welche Sicherheitsmechanismen von iOS umgangen werden müssen, damit ein JB funktioniert. \par 
Im wissenschaftlichen Teil der Arbeit versucht einen Zusammenhang zwischen den JBs und der Sicherheit der iOS Produkte herzustellen. Die erhoben Daten werden mit der Goal-Question-Metrik (GQM) Methode aufbereitet. Die daraus resultieren Metriken werden verwendet, um die zuvor erarbeiteten Fragen (Question) zu beantworten. Diese Antworten geben Auskunft inwieweit, dass zuvor definiert Ziel (Goal) erreicht wurde oder nicht. \par 
In dieser Arbeit wird die Hypothese aufgestellt, dass die Anzahl der Tagen, die die Jailbreak-Community benötigt, um einen Jailbreak zu veröffentlichen, eine Aussage zulässt, wie sicher und vertrauenswürdig diese iOS Version ist.

% --- English Abstract ----------------------------------------------------

\cleardoublepage

%\selectlanguage{english}

\begin{center}
{\Large\bfseries Abstract}
\end{center}

(E.g. ``This thesis deals with...'')

%\selectlanguage{austrian}
