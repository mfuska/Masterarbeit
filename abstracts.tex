%----------------------------------------------------------------
%
%  File    :  abstracts.tex
%
%  Authors :  Keith Andrews, IICM, TU Graz, Austria
%             Manuel Koschuch, FH Campus Wien, Austria
% 
%  Created :  22 Feb 96
% 
%  Changed :  30 Oct 2008
% 
%----------------------------------------------------------------


% --- German and English Abstracts ------------------------------------------------

% --- German Abstract ----------------------------------------------------
\cleardoublepage

\begin{center}
{\Large\bfseries Kurzfassung}
\end{center}
In dieser Arbeit werden die Sicherheit und Vertrauenswürdigkeit der iOS Produkte untersucht. Die Sicherheitsmechanismen der iOS Produkte werden in historischer Ordnung angeführt. Im Detail wird die iOS Hardware und Software Architektur beschrieben, weiters werden die wichtigsten iOS Sicherheitsmechanismen beschrieben und die Zusammenhänge werden erklärt. \par 
Ein weiterer Teil dieser Arbeit setzt sich mit den Jailbreak von iOS Devices/Versionen auseinander. Es wird angeführt, welche iOS Sicherheitsmechanismen umgangen werden müssen, damit ein JB erfolgreich durchgeführt werden kann. \par 
Im wissenschaftlichen Teil der Arbeit wird versucht, einen Zusammenhang zwischen den JBs und der Sicherheit der iOS Produkte herzustellen. Die Hypothese, die in dieser Arbeit aufgestellt wurde, wird anhand der erhoben Rohdaten auf ihrer Richtigkeit überprüft. \par
Diese Daten werden mit Hilfe der Goal-Question-Metrik (GQM) Methode aufbereitet. Die daraus resultieren Metriken werden verwendet, um die zuvor erarbeiteten Fragen (Question) zu beantworten. Die daraus abgeleiteten Antworten geben Auskunft inwieweit das zuvor definiert Ziel (Goal) erreicht wurde.\par 

\paragraph{Goal G1:}\textit{\glqq Das Ziel dieser Arbeit ist es eine Aussage über die Sicherheit und die Vertrauenswürdigkeit der iOS Produkte(iPhone, iPad) tätigen zu können.\grqq{}} \par
\textbf{Goal G1.1:} \textit{\glqq Erstes abgeleitetes Ziel ist es einen Zusammenhang zwischen der Sicherheit und der Vertrauenswürdigkeit des iOS Device und der verwenden Hardware der iOS Geräte herzustellen.\grqq{}} \par 
\textbf{Goal G1.2:} \textit{\glqq Zweites abgeleitetes Ziel dieser Arbeit ist es einen Zusammenhang zwischen der Sicherheit und der Vertrauenswürdigkeit des iOS Device und der installierten iOS Software herzustellen.\grqq{}}

\paragraph{Question Q1:} \textit{\glqq Welche Faktoren sind für die Sicherheit und die Vertrauenswürdigkeit eines iOS Device ausschlaggebend?\grqq{}}
\paragraph{Question Q2:} \textit{\glqq Welche Auswirkung haben die von Apple eingeführten Sicherheitsmechanismen und Sicherheitsupdates auf die Sicherheit des Systems?\grqq{}}        
\paragraph{Hypothese H1:}\textit{\glqq Die Anzahl der Tage, die die JB Community benötigt, um ein JB für eine iOS Version oder ein iOS Device zu veröffentlichen, kann als Indikator für die Sicherheit und Vertrauenswürdigkeit der iOS Produkte herangezogen werden.\grqq{}}



% --- English Abstract ----------------------------------------------------

\cleardoublepage

%\selectlanguage{english}

\begin{center}
{\Large\bfseries Abstract}
\end{center}

(E.g. ``This thesis deals with...'')

%\selectlanguage{austrian}
