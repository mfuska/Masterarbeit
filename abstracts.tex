%----------------------------------------------------------------
%
%  File    :  abstracts.tex
%
%  Authors :  Keith Andrews, IICM, TU Graz, Austria
%             Manuel Koschuch, FH Campus Wien, Austria
% 
%  Created :  22 Feb 96
% 
%  Changed :  30 Oct 2008
% 
%----------------------------------------------------------------


% --- German and English Abstracts ------------------------------------------------

% --- German Abstract ----------------------------------------------------
\cleardoublepage

\begin{center}
{\Large\bfseries Kurzfassung}
\end{center}
In dieser Arbeit werden die Sicherheit und Vertrauenswürdigkeit der iOS Produkte untersucht. Die Sicherheitsmechanismen der iOS Produkte werden in historischer Ordnung angeführt. Im Detail wird die iOS Hardware- und Software-Architektur beschrieben, weiters werden die wichtigsten iOS Sicherheitsmechanismen beschrieben und die Zusammenhänge erklärt. \par 
Ein weiterer Teil dieser Arbeit setzt sich mit den Jailbreaks von iOS Devices/Versionen auseinander. Es wird angeführt, welche iOS Sicherheitsmechanismen umgangen werden müssen, damit ein JB erfolgreich durchgeführt werden kann. \par 
Im wissenschaftlichen Teil der Arbeit wird versucht, einen Zusammenhang zwischen den JBs und der Sicherheit der iOS Produkte herzustellen. Die Hypothese, die in dieser Arbeit aufgestellt wurde, wird anhand der erhobenen Rohdaten auf ihrer Richtigkeit überprüft. \par
Diese Daten werden mit Hilfe der Goal-Question-Metrik (GQM) Methode aufbereitet. Die daraus resultierenden Metriken werden verwendet, um die zuvor erarbeiteten Fragen (Questions) zu beantworten. Die daraus abgeleiteten Antworten geben Auskunft inwieweit das zuvor definierte Ziel (Goal) erreicht wurde.\par 

\paragraph{Goal G1:}\textit{\glqq Das Ziel dieser Arbeit ist es, eine Aussage über die Sicherheit und die Vertrauenswürdigkeit der iOS Produkte (iPhone) tätigen zu können.\grqq{}} \par
\textbf{Goal G1.1:} \textit{\glqq Erstes abgeleitetes Ziel ist es, einen Zusammenhang zwischen der Sicherheit und der Vertrauenswürdigkeit des iOS Device und der verwendenten Hardware der iOS Geräte herzustellen.\grqq{}} \par 
\textbf{Goal G1.2:} \textit{\glqq Zweites abgeleitetes Ziel dieser Arbeit ist es, einen Zusammenhang zwischen der Sicherheit und der Vertrauenswürdigkeit des iOS Device und der installierten iOS Software herzustellen.\grqq{}}

\paragraph{Question Q1:} \textit{\glqq Welche Faktoren sind für die Sicherheit und die Vertrauenswürdigkeit eines iOS Device ausschlaggebend?\grqq{}}
\paragraph{Question Q2:} \textit{\glqq Welche Auswirkung haben die von Apple eingeführten Sicherheitsupdates auf die Sicherheit des Systems?\grqq{}}        
\paragraph{Hypothese H1:}\textit{\glqq Die Anzahl der Tage, die die JB Community benötigt, um ein JB für eine iOS Version oder ein iOS Device zu veröffentlichen, kann als Indikator für die Sicherheit und Vertrauenswürdigkeit der iOS Produkte herangezogen werden.\grqq{}}



% --- English Abstract ----------------------------------------------------

\cleardoublepage

%\selectlanguage{english}

\begin{center}
{\Large\bfseries Abstract}
\end{center}
This thesis investigates the security and trustability of the iOS products. The iOS hardware and software will be described hereby in detail as well as the most important iOS security mechanisms, which will be listed in historical order. Furthermore their associations will be explained.  \par
Another part will investigate the jailbreak of iOS devices and iOS versions. It will be mentioned which iOS security measures have to be bypassed in order to achieve a successful jailbreak. \par 
In the scientific section the author attempts to create a context between the jailbreaks and the security of the iOS products. The hypothesis which is raised in that work will be verified based on upraised raw data.  \par
Those data are derived by means of the goal-question-metric (GQM) method. The resulting metrics are used to answer the previously constructed questions. Thereby deducted answers indicate to what extend the previously defined goal was achieved.  \par

\paragraph{Goal G1:} \textit{\glqq The goal of this thesis is to provide a statement about the security and the trustability of  iOS products (such as iPhone).\grqq{}} \par
\textbf{Goal G1.1: }\textit{\glqq The first deducted goal is to create a context between the security and trustability of iOS devices and their hardware.\grqq{}} \par
\textbf{Goal G1.2:} \textit{\glqq The second deducted goal is to create a context between the security and trustability of an iOS product and the installed iOS software.\grqq{}}\par

\paragraph{Question Q1:} \textit{\glqq Which factors are crucial for the security and trustability of an iOS device?\grqq{}} \par
\paragraph{Question Q2:} \textit{\glqq Which implications do the security measures and security updates introduced by Apple have in respect of the security of the system?\grqq{}} \par

\paragraph{Hypothesis H1:} \textit{\glqq The amount of days which are required by the JB community to publish a JB for an iOS version or an iOS device might be used as an indicator for the security and trustability of an iOS product. \grqq{}} \par

%\selectlanguage{austrian}
