%----------------------------------------------------------------
%
%  File    :  glossary.tex
%
%  Author  :  Manuel Koschuch, FH Campus Wien, Austria
% 
%  Created :  30 Oct 2008
% 
%  Changed :  30 Oct 2008
% 
%----------------------------------------------------------------

\begin{center}
{\Large\bfseries Begriffsdefinition Verzeichnis}
\end{center}

\begin{table*}[htbp]
    \begin{center}
      \begin{tabular}{p{3cm}p{12cm}} 
        
        \textbf{App Store} &  Der Begriff App Store bezeichnet einen Online Store, welcher es ermöglicht Apps herunterzuladen und zu installieren\\ 
        
        \textbf{Backdoor} &  Eine Backdoor bezeichnet einen Teil einer Software, der es dem User ermöglicht, unter Umgehung der normalen Zugriffssicherung Zugang zum iOS Device zu erhalten.\\
        \textbf{Bug} &  Als Bug wird ein Fehlverhalten von Computerprogrammen bezeichnet\\ 
		
		\textbf{Cleartext} &  Als Cleartext werden Daten bezeichnet die ohne weitere Bearbeitung (entschlüsseln) verarbeitet/gelesen werden können\\     
		
		\textbf{Framework} & Ein Framework ist ein Programmiergerüst, das im Rahmen der objektorientierten Softwareentwicklung verwendet wird. \\
		
		\textbf{Hacker} &  Der Begriff Hacker wird verwendet wenn eine Person oder ein System versucht Zugriff auf ein fremdes Systemen unerlaubt zu erhalten.\\
		\textbf{Hook} & Der Begriff Hook beschreibt in der Softwareentwicklung eine Programmschnittstelle.\\
		  
		\textbf{iOS} &  Der Begriff iOS steht für das mobile Apple Betriebssystem\\
		  
		 \textbf{Jailbreak} &  Als Jailbreak wird, dass nicht-autorisierte Entfernen von Nutzungsbeschränkungen bei iOS Geräten bezeichnet.\\  
		 \textbf{Kernelspace} & Der Begriff Kernelspace beschreibt einen Bereich im virtuellen Memory, in dem nur privilegierte Prozess des Betriebssystems laufen.\\ 
		      
		  \textbf{Malware} &  Als Malware wird eine Schadsoftware bezeichnet, die unerwünschte bzw. schädliche Programme ausführt.\\
		 
		  \textbf{proc-Struktur} & Ist ein virtuelles Filesystem in dem alle \textit{\glqq runtime\grqq{}} Prozess Informationen gespeichert werden.\\
		  
		  \textbf{Userspace} & Der Begriff Userspace beschreibt einen Bereich im virtuellen Memory, in dem auch nicht privilegierte Prozess des Betriebssystems laufen.\\
		  
		 \textbf{Root-Rechte} &  Der Begriff Root-Recht steht für den uneingeschränkten Zugriff auf das Betriebssystem und auf die Device Ressourcen.\\
		 \textbf{Root-User} &  Als Root-User bezeichnet man den User der uneingeschränkten Zugriff auf das Betriebssystem und auf die Device Ressourcen.\\ 
		  
		 \textbf{Zero Day Bug} &  Der Begriff Zero Day Bug steht für einen Softwarefehler der bis jetzt noch nicht veröffentlicht wurde und den Hackern noch zur Verfügung steht.\\
		  
		\end{tabular}		    
    \end{center}
\end{table*}