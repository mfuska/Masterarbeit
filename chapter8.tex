%----------------------------------------------------------------
%
%  File    :  chapter8.tex
%
%  Authors : Michael Fuska, FH Campus Wien, Austria% 
%  Created : 13 Feb 2016
%
%  Changed :  
% 
%----------------------------------------------------------------


\chapter{Conclusion}
\label{ch:Conclusion}

Wie uns der Fall \textit{\glqq FBI gegen Apple\grqq{}} zeigt, war die Implementierung der Secure Enclave ein massiver Sicherheitsgewinn für die iOS Produkte. Unter iOS gibt es die Möglichkeit neben der Konfiguration des Passcodes, auch die Möglichkeit, dass nach zehn falschen Eingaben des Passcodes alle Daten des Devices gelöscht werden. Der Wert der Fehlversuche wurde bis zum Apple A7 Prozessors im Flash Memory geschrieben, dadurch war es möglich den Wert immer wieder zurückzusetzen, bis eine gültige Passcode Eingabe erfolgt. Ab dem Apple A7 Prozessor ist dies nicht mehr möglich, da der Wert der Fehlversuche in der Secure Enclave gespeichert werden und somit dieser Wert nicht mehr überschrieben werden kann. 

Aussage tätigen über die Hypothese.

