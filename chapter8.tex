%----------------------------------------------------------------
%
%  File    :  chapter8.tex
%
%  Authors : Michael Fuska, FH Campus Wien, Austria% 
%  Created : 13 Feb 2016
%
%  Changed :  
% 
%----------------------------------------------------------------

\chapter{Conclusion}
\label{ch:Conclusion}
Apple versucht die  Anforderungen \textit{\glqq usability\grqq{}} und \textit{\glqq security\grqq{}} in ihren iOS Produkten zu vereinen. Das ist eine der Begründungen für die restriktiven Sicherheitsmechanismen von Apple. Die iOS Produkte sind ein geschlossenes System. Dies bedeutet, dass weder die Hardware noch die Software vom User verändert werden können. Dies ist auch ein Grund für die Stabilität der iOS Produkte. Ein weiterer Grund ist die simple Konfiguration des iOS Systems. Der User benötigt keine Vorkenntnisse, um eine sichere iOS Konfiguration vorzunehmen. Die Default-Konfiguration des iOS Device ist als sicher anzusehen. Nur die Passcode Konfiguration sollte von vier Digits auf sechs Digits verändert werden, da die gesamte Sicherheit des Systems von dem verwendeten Passcode abhängt, inklusive der Datenverschlüsselung des iOS Gerätes. \par 

\begin{description}
    \item[\parbox{\textwidth} {Die Sicherheit und Vertrauenswürdigkeit eines mobilen Gerätes hängt von mehreren Faktoren ab}]~\par
    \begin{enumerate}
        \item Dem Passcode, der zum Entsperren des Gerätes verwendet wird.
        \item Den Apps, die auf dem Gerät installiert wurden. 
        \item Der Verschlüsselung, die verwendet wird, um die Daten des Gerätes zu verschlüsseln.
        \item Den Sicherheitsmechanismen, die im mobilen Betriebssystems umgesetzt wurden.
        \item Den regelmäßigen Sicherheitsupdates, die dazu verwendet wurden, um die Sicherheitslücken des mobilen Betriebssystems zu schließen.  
    \end{enumerate}
\end{description} 
Alle diese fünf Punkte werden von den iOS Produkten unterstützt. Installiert der User auf seinem iOS Device ein JB, so verlässt der User freiwillig die sichere iOS Umgebung. Die Installation eines JBs hat einen massiven Einfluss auf die Sicherheit und die Vertrauenswürdigkeit des iOS Systems, da die Sicherheitsmechanismen von Apple umgangen und/oder deaktiviert werden.\par  
Ein JB erlaubt es, dass Apps mit einem selbst signierten Zertifikat auf dem iOS Device installiert werden können. Es liegt in der Hand des Users abzuschätzen, inwieweit er dem Hersteller der Apps vertraut, die er auf seinem iOS Device installieren möchte. Diese Apps wurden nicht auf Malware geprüft, und somit ist die Installation der Apps ein Glücksspiel. Sicherheitsanalysen zeigen, dass die meisten Viren auf einem iOS Device nur funktionieren, wenn ein JB auf dem Gerät installiert wurde. Ein Beispiel dafür ist die Malware KeyRaider \cite{KeyRaider}. Diese Malware funktioniert nur auf einem \textit{\glqq gejailbreaked\grqq{}} iOS Device und wird mit einer App installiert, die über Cydia heruntergeladen wurde. \par 
 Ein weiterer Punkt der in diesem Zusammenhang angeführt werden muss, ist die Tatsache, dass die Konfiguration mancher Apps aus dem nicht offiziellen App Store Fachwissen benötigt. Der User muss sich bewusst sein, welche Seiteneffekte diese Installation haben kann. Nach der Installation eines JBs haben alle iOS Devices dasselbe root-Passwort und der User muss dieses neu setzen. Der Virus \textit{\glqq ikee\grqq{}} basiert genau darauf, dass der User sein root-Passwort nicht geändert hat. Der Virus hat sich am iOS Device angemeldet und persönliche Daten manipuliert.\par 
 Im mobilen Betriebssystem von Apple sind die Sicherheitsupdates fest verankert. Der User kann die Sicherheitsupdate-Benachrichtigungen des iOS nicht deaktivieren. Wenn der User ein JB auf seinem iOS Device installiert hat, wird er nicht alle iOS Sicherheitsupdates auf seinem Gerät installieren, da der JB mit einem Sicherheitsupdate verloren gehen würde.\par 
Das ist die eine Seite des JBs. Die andere Seite ist, dass Apple durch \textit{\glqq Reverse Engineering\grqq{}} der JB-Tools die Sicherheitslücken, die von den JBs ausgenutzt werden, durch iOS Sicherheitsupdates schließt. Die Analyse der iOS Sicherheitsupdates zeigt, dass einige Sicherheitslücken nur auf Grund der JBs und der Begutachtung dieses bekannt geworden sind. Apple ist Aufgrund des öffentlichen Drucks \textit{\glqq gezwungen\grqq{}}, die Bugs der iOS Versionen zeitnahe zu schließen. Dieser Druck entsteht dadurch, dass durch die veröffentlichten JBs den iOS Kunden bewusst wird, dass ihr iOS Device nicht sicher ist und sie nicht darauf vertrauen können, dass ihre Privatsphäre gewahrt bleibt.


