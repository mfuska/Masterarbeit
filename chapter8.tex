%----------------------------------------------------------------
%
%  File    :  chapter8.tex
%
%  Authors : Michael Fuska, FH Campus Wien, Austria% 
%  Created : 13 Feb 2016
%
%  Changed :  
% 
%----------------------------------------------------------------


\chapter{Conclusion}
\label{ch:Conclusion}
Die Apple Produkte sind eine Gratwanderung zwischen \textit{\glqq usability\grqq{}} und \textit{\glqq security\grqq{}}. Dies ist der Hauptgrund für die User um ein JB auf dem iOS Device zu installieren. Hat der User auf einem iOS Device ein JB installiert, so verlässt der User freiwillig die sicher iOS Umgebung. Es hat einen  massiven Einfluss auf die Sicherheit und auf die Vertrauenswürdigkeit des iOS Produktes, wenn der User ein JB installiert. Ein JB erlaubt es, dass Apps mit einen selbst signierten Zertifikat auf dem iOS Device installiert werden kann. Es liegt in der Hand des User abzuschätzen oder er dem Hersteller der App soweit vertraut, dass er die App am Device installiert. Diese Apps wurde nicht auf Malware geprüft und somit ist die Installation dieser App ein Glücksspiel. Sicherheitsanalysen zeigen, dass die meisten Viren nur auf einem iOS Device funktionieren, die ein JB installiert haben.\par  
Das ist die eine Seite des JBs, die andere Seite ist, dass durch \textit{\glqq Reverse Engineering\grqq{}} der JB-Tools die Sicherheitslücken in der iOS Software geschlossen werden. Die Sicherheitsupdates zeigen, dass einige Sicherheitslücken nur auf Grund der JBs geschlossen werden. Da Apple durch den öffentlichen Druck \textit{\glqq gezwungen\grqq{}} wird die Bugs der iOS Versionen zuschliessen. Dieser Druck entsteht dadurch, das für die Öffentlichkeit die Unsicherheit des iOS Devices mit einem JB \textit{\glqq sichtbar\grqq{}} gemacht wird. 

