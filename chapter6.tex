%----------------------------------------------------------------
%
%  File    :  chapter6.tex
%
%  Authors : Michael Fuska, FH Campus Wien, Austria% 
%  Created : 13 Feb 2016
%
%  Changed :  
% 
%----------------------------------------------------------------


\chapter{Analyse und Ergebnisse}
\label{ch:Ergebnisse}

%------------------------------------------------------------------------------
%------------------------------ Analysen zur Frage 1

\section{Analysen zur Frage 1}
\label{sec:Frage1}

\textit{\glqq Q1: Welche Faktoren sind für die Sicherheit und die Vertrauenswürdigkeit eines iOS Device ausschlaggebend \grqq{}}

\begin{table}[htp!]
    \begin{center}
        \begin{tabular}{|p{30mm}|p{27mm}|p{12mm}|p{18mm}|p{2cm}|p{22mm}|} \hline
            \textbf{iOS Device} & \textbf{Verkaufsstart} & \textbf{initial iOS} & \textbf{Secure Enclave} & \textbf{Prozessor}  & \textbf{\#Tage JB} \\ \hline
            \textbf{iPhone} & 29.06.2007  & 1.0 & nA & Samsung SSL8900 & 11\\ \hline
            \textbf{iPhone 3G} & 11.07.2008 & 2.0 & nA & Samsung SSL8900 & 9\\ \hline
            \textbf{iPhone 3GS} & 19.06.2009 & 3.0 & nA & Samsung SSL8920 & 14\\ \hline
            \textbf{iPhone 4} & 01.08.2010 & 4.0 & nA & Apple A4 & 38 \\ \hline
            \textbf{iPhone 4s} & 20.01.2012 & 5.0 & nA & Apple A5 & 98 \\ \hline 
            \textbf{iPhone 5} & 21.09.2012 & 6.0 & nA & Apple A6 & 136 \\ \hline
            \textbf{iPhone 5c} & 22.12.2013 & 7.0 & nA & Apple A6 & 93 \\ \hline
            \textbf{iPhone 5s} & 22.12.2013 & 7.0 & A & Apple A7 & 93 \\ \hline
            \textbf{iPhone 6} & 19.09.2014 & 8.0 & A & Apple A8 & 33\\ \hline
            \textbf{iPhone 6 Plus} & 19.09.2014 & 8.0 & A & Apple A8 & 33\\ \hline
            \textbf{iPhone 6s} & 25.09.2016 & 9.0 & A & Apple A9 & 19\\ \hline
            \textbf{iPhone 6s Plus} & 25.09.2016 & 9.0 & A & Apple A9 & 19\\ \hline
            \textbf{iPhone SE} & 31.03.2016 & 9.0 & A & Apple A9 & nA\\ \hline  
        \end{tabular} 
        \caption{Auflistung iOS Device/ Verkaufsstart/ initiale iOS/ Prozessor/ \#Tage bis zum JB}
        \label{tab:iOSHW}
    \end{center}
\end{table}
In der Tabelle \ref{tab:iOSHW} werden alle iOS Devices aufgelistet und in Abhängigkeit von den Verkaufszeitpunkt, iOS Auslieferungsversion und des  Prozessors Typ des iDevice gebracht. Weiters wird auch angeführt, ob dieses Device einen Koprozessor mit Secure Enclave implementiert hat und wie lange die JB Community benötigte, um ein JB auf diesem Device zu installieren.\par
Die Tabelle \ref{tab:iOSHW} alleine gibt noch keinen Aufschluss über den Zusammenhang zwischen der Sicherheit des Systems und der verwendeten iOS Hardware. Aber im Zusammenhang mit der installierten iOS Version können Rückschlüsse über die Sicherheit und Vertrauenswürdigkeit des Systems getroffen werden.

\begin{figure}[htbp]
        \centering
                \includegraphics[scale=0.6]{Bilder/iDeviceJB-SW-HW.png}
         \caption{Vergleich der Anzahl der Tage eines JB für die Prozessoren und für der initialen iOS Version}
        \label{tab:VergleichJBProzessorSW}      
\end{figure}


%\begin{table}[htp!]
%    \begin{center}
%        \begin{tabular}{|p{30mm}|p{12mm}|p{18mm}|p{18mm}|} \hline
%            \textbf{iOS Device} & \textbf{initial iOS} & \textbf{\#Tage JB HW} & \textbf{\#Tage JB SW} \\ \hline
%            \textbf{iPhone} & 1.0 & 11 & 11\\ \hline
%            \textbf{iPhone 3G} & 2.0 & 9 & 9\\ \hline
%            \textbf{iPhone 3GS} & 3.0 & 14 & 2\\ \hline
%            \textbf{iPhone 4} & 4.0 & 38 & 2\\ \hline
%            \textbf{iPhone 4s} & 5.0 & 98 & 1\\ \hline 
%            \textbf{iPhone 5} & 6.0 & 136 & 0\\ \hline
%            \textbf{iPhone 5c} & 7.0 & 93 & 95\\ \hline
%            \textbf{iPhone 5s} & 7.0 & 93 & 95\\ \hline
%            \textbf{iPhone 6} & 8.0 & 33 & 35\\ \hline
%            \textbf{iPhone 6 Plus} & 8.0 & 33 & 35\\ \hline
%            \textbf{iPhone 6s} & 9.0 & 19 & 28\\ \hline
%            \textbf{iPhone 6s Plus} & 9.0 & 19 & 28\\ \hline
%            \textbf{iPhone SE} & 9.0 & nA & nA \\ \hline  
%        \end{tabular} 
%        \caption{Auflistung iOS Device/ Initiale iOS/ \#Tage bis zum HW JB / \#Tage bis zum SW JB}
%        \label{tab:AuflistungDeviceHWiOS}
%    \end{center}
%\end{table}


G1.2: \textit{\glqq Zweites abgeleitetes Ziel dieser Arbeit ist es einen Zusammenhang zwischen der Sicherheit und der Vertrauenswürdigkeit des iOS Device und der installierten iOS Software herzustellen?\grqq{}}
\begin{table}[htp!]
    \begin{center}
        \begin{tabular}{|l|l|l|} \hline
         \textbf{iOS Version} & \textbf{Veröffentilicht} & \textbf{Tage JB}\\ \hline    
        1.0 & 29.06.2007 & 11\\ \hline 
        2.0 & 11.07.2008	& 9\\ \hline 
        3.0 & 17.06.2009	& 2\\ \hline 
        4.0 & 21.06.2010 & 2\\ \hline 
        5.0 & 12.10.2011	& 1\\ \hline 
        6.0 & 19.09.2012	& 0\\ \hline 
        7.0 & 18.09.2013	& 95\\ \hline 
        7.1-7.1.2 & 29.05.2014 & 25\\ \hline 
        8.0 & 17.09.2014	& 35\\ \hline 
        8.1.1-8.4 & 17.11.2014	& 12\\ \hline 
        9.0 & 16.09.2015	& 28\\ \hline
       %  9.0.1 & 23.09.2015 & nA \\ \hline
       %  9.0.2 & 30.09.2015 & nA \\ \hline 
        9.1 & 21.10.2015	& 142\\ \hline 
       %  9.2 & 08.12,2015 & nA\\ \hline
       % 9.2.1 & 18.02.2016 & nA \\ \hline
       %  9.3 & 21.03,016 & nA\\ \hline 
       % 9.3.1 & 31.03.2016 & nA\\ \hline
       % 9.3.2 & 17.05.2016 & nA \\ \hline
        \end{tabular} 
        \caption{Sicherheitszusammenhang iOS Version und JB}
        \label{tab:iOSVersion}
    \end{center}
\end{table}
         
\begin{figure}[htbp]
        \centering
                \includegraphics[height=11cm]{Bilder/Frage1_1.png}
        \caption{iOS Version / JB / Tage}
        \label{fig:AnalyseiOSJB1}        
\end{figure}

\begin{figure}[htbp]
        \centering
                \includegraphics[height=10cm]{Bilder/Frage1_2.png}
        \caption{iOS Version / JB / Tage}
        \label{fig:AnalyseiOSJB2}
\end{figure}

\section{Frage 2}
\label{sec:Frage2}
\textit{\glqq Welche Auswirkung haben die von Apple eingeführten Sicherheitsmechanismen und Sicherheitsupdates auf die Sicherheit des Systems?\grqq{}} 
            
     
\begin{description}
    \item[\parbox{\textwidth} {Antwort kurz INFO Katharina}]~\par
        \begin{itemize}
                \item Zuordnung Bugs / Sicherheitsmechanismen Ausreißer in den Stats Veröffentlichung \#Tage JB  
        \end{itemize}
\end{description} 
        




 