%----------------------------------------------------------------
%
%  File    :  chapter6.tex
%
%  Authors : Michael Fuska, FH Campus Wien, Austria
%  Created : 13 Feb 2016
%
%  Changed :  
% 
%----------------------------------------------------------------


\chapter{Analyse und Ergebnisse}
\label{ch:Ergebnisse}

%------------------------------------------------------------------------------
%------------------------------ Analysen zur Frage 1

\section{Welche Faktoren sind für die Sicherheit und die Vertrauenswürdigkeit eines iOS Device ausschlaggebend}
\label{sec:Frage1}

\begin{table}[htp!]
    \begin{center}
        \begin{tabular}{|p{30mm}|p{27mm}|p{12mm}|p{10mm}|p{18mm}|p{2cm}|p{15mm}|} \hline
            \textbf{iOS Device} & \textbf{Verkaufsstart} & \textbf{initial iOS} & \textbf{last iOS} & \textbf{Secure Enclave} & \textbf{Prozessor}  & \textbf{\#Tage JB} \\ \hline
            \textbf{iPhone} & 29.06.2007  & 1.0 & 3.1.3 & nA & Samsung SSL8900 & 11\\ \hline
            \textbf{iPhone 3G} & 11.07.2008 & 2.0 & 4.2.1 & nA & Samsung SSL8900 & 9\\ \hline
            \textbf{iPhone 3GS} & 19.06.2009 & 3.0 & 6.1.6 & nA & Samsung SSL8920 & 14\\ \hline
            \textbf{iPhone 4} & 01.08.2010 & 4.0 & 7.1.2 & nA & Apple A4 & 38 \\ \hline
            \textbf{iPhone 4s} & 20.01.2012 & 5.0 & 9.3.2 & nA & Apple A5 & 98 \\ \hline 
            \textbf{iPhone 5} & 21.09.2012 & 6.0 &  9.3.2 & nA & Apple A6 & 136 \\ \hline
            \textbf{iPhone 5c} & 22.12.2013 & 7.0 & 9.3.2 & nA & Apple A6 & 93 \\ \hline
            \textbf{iPhone 5s} & 22.12.2013 & 7.0 & 9.3.2 & A & Apple A7 & 93 \\ \hline
            \textbf{iPhone 6} & 19.09.2014 & 8.0 & 9.3.2 & A & Apple A8 & 33\\ \hline
            \textbf{iPhone 6 Plus} & 19.09.2014 & 8.0 & 9.3.2 &  A & Apple A8 & 33\\ \hline
            \textbf{iPhone 6s} & 25.09.2016 & 9.0 &  9.3.2 & A & Apple A9 & 19\\ \hline
            \textbf{iPhone 6s Plus} & 25.09.2016 & 9.0 & 9.3.2 &  A & Apple A9 & 19\\ \hline
            \textbf{iPhone SE} & 31.03.2016 & 9.0 &  9.3.2 & A & Apple A9 & nA\\ \hline  
        \end{tabular} 
        \caption{Auflistung iOS Device/ Verkaufsstart/ initiale iOS/ last supported iOS / Prozessor/ \#Tage bis zum JB}
        \label{tab:iOSHW}
    \end{center}
\end{table}

In der Tabelle \ref{tab:iOSHW} werden alle iOS Devices aufgelistet und  in Abhängigkeit vom Verkaufszeitpunkt des Device, der initial installierten iOS Version und des Prozessor-Typs des iDevice gebracht. Weiters wird in der Tabelle angeführt, ob dieses Device einen Koprozessor mit Secure Enclave implementiert hat und wie lange die JB Community benötigte, um ein JB für dieses Device zu veröffentlichen. Diese Tabelle beinhaltet alle Daten die als Grundlage für alle weiteren Analysen in diesem Kapitel dienen.\par

%------------------------------------------------------------------------------
%------------------------------ 
\subsection{Secure Enclave}
\label{sec:Frage1SecureEnclave}
 
 Unter iOS gibt es die Möglichkeit verschiedene Passcode Konfigurationen vorzunehmen. Neben der Anzahl der Stellen, kann auch konfiguriert werden, ob nur Zahlen oder auch alphanumerische Werte für einen Passcode verwendet werden.\par 
 Ein weiterer iOS Konfigurationsparameter ermöglicht es, das Device so zu konfigurieren, dass nach zehn falschen Passcode Eingabe alle Daten des Gerätes gelöscht werden. Die Anzahl der Fehlversuche wird bei iOS Devices ohne Secure Enclave in den Flash Memory geschrieben. Bei iOS Devices mit Secure Enclave werden diese Daten in der \textit{\glqq ARM Trust Zone\grqq{}} gespeichert und können somit nicht ohne weiters ausgelesen und überschrieben werden. 
 
 Die Secure Enclave des Koprozessor (siehe Tabelle: \ref{tab:iOSHW}) bringen einen massiven Sicherheitsgewinn für die iOS Produkte mit sich, da die  
 
 Im Fall \textit{\glqq FBI gegen Apple\grqq{}} wurde der Tatsache, dass das iOS Device keine Secure Enclave enthält, besonderer Bedeutung beigemessen. Das FBI beauftragte ein Unternehmen um das iPhone 5c des Attentäters zu entsperren. Die Abbildung: \ref{fig:iOSSecurityArchitekturiOS7} zeigt die iOS Sicherheitsarchitektur des iPhone 5c. Die Systemarchitektur dieses iOS Device beinhaltet keine Secure Enclave. \par 
\begin{description}
    \item[\parbox{\textwidth} { Der Sicherheitsforscher Zdziarski beschreibt in seinen Abhandlungen die plausiblen Varianten des FBI Hacks wie folgt}]~\par
    \begin{enumerate}
        \item \textit{\glqq .... based on all of this, is that an external forensics company, with hardware capabilities, is likely copying the NAND storage off the chip and frequently re-copying all or part of the chip’s contents back to the device in order to brute force the pin – and may or may not also be using older gear from iOS 8 techniques to do it. The two weeks the FBI has asked for are not to develop this technique (it’s most likely already been developed, if FBI is willing to vacate a hearing over it), but rather to demonstrate, and possibly sell, the technique to FBI by means of a field test on some demo units.\grqq{}} \cite{Hacking[4]}
        \item \textit{\glqq If the FBI did in fact use a software exploit, the question then becomes one of how viable it is on other platforms.\grqq{}} \cite{Hacking[4]}
    \end{enumerate}
\end{description} 

\paragraph{Ergebnisse:} Die Secure Enclave biete einen Sicherheitsgewinn für die iOS Produkte, aber der Fall \textit{\glqq FBI via Apple\grqq{}} lässt einige Fragen offen. Da das FBI, Apple die Sicherheitslücke nicht offenlegte gibt es nur Mutmaßungen über den Hack der verwendet wurde um das iPhone des Attentäters zu entsperren. \par 

Zdziarski beschreibt die möglichen Folgen dieses Hacks wie folgt: \textit{\glqq The moral of the story is that the exploit the FBI may have is dangerous in and of itself, regardless of whether it serves their specific purposes of brute forcing a device’s pin. Such an exploit has numerous uses within the intelligence community and poses a threat to not only the hundreds of millions of older devices out there, but if it can be ported to a 64-bit platform, every single one of us – either directly as a threat from the government, a nation state the exploit developer also sold it to, or another hacker who finds the same hole because FBI didn’t report the vulnerability to Apple. FBI has left us all potentially exposed by choosing to keep their technique secret.\grqq{}} \cite{Hacking[4]} \par 

Vor allem gibt es Vermutungen, dass der Hack des FBIs eine Möglichkeit bietet die Secure Enclave zu umgehen.

%------------------------------------------------------------------------------
%------------------------------ 
\subsection{iOS Device}
\label{sec:Frage1iOSDevice} 

Die Tabelle \ref{tab:iOSHW} alleine gibt noch keinen Aufschluss über den Zusammenhang zwischen der Sicherheit des Systems und der verwendeten iOS Hardware. Da die Tabelle \ref{tab:iOSHW} keine Aussagen darüber zulässt, ob der Zeitraum bis zum Veröffentlichen des JB nur von der HW abhängt oder von der iOS Version die am Device installiert wurde, müssen die Daten der beiden Tabellen (Tabelle: \ref{tab:iOSHW}, Tabelle: \ref{tab:iOSVersion}) korreliert werden. \par 
Die Abbildung \ref{fig:VergleichJBProzessorSW} stellt die Prozessoren des iDevices mit der installierten iOS Version und die benötigten Tage für die Veröffentlichung eines JB in Verbindung. \par


\begin{table}[htp!]
    \begin{center}
        \begin{tabular}{|l|l|l|} \hline
         \textbf{iOS Version} & \textbf{Veröffentilicht} & \textbf{\#Tage JB}\\ \hline    
        1.0 & 29.06.2007 & 11\\ \hline 
        2.0 & 11.07.2008	& 9\\ \hline 
        3.0 & 17.06.2009	& 2\\ \hline 
        4.0 & 21.06.2010 & 2\\ \hline 
        5.0 & 12.10.2011	& 1\\ \hline 
        6.0 & 19.09.2012	& 0\\ \hline 
        7.0 & 18.09.2013	& 95\\ \hline 
        7.1-7.1.2 & 29.05.2014 & 25\\ \hline 
        8.0 & 17.09.2014	& 35\\ \hline 
        8.1.1-8.4 & 17.11.2014	& 12\\ \hline 
        9.0 & 16.09.2015	& 28\\ \hline
       %  9.0.1 & 23.09.2015 & nA \\ \hline
       %  9.0.2 & 30.09.2015 & nA \\ \hline 
        9.1 & 21.10.2015	& 142\\ \hline 
       %  9.2 & 08.12,2015 & nA\\ \hline
       % 9.2.1 & 18.02.2016 & nA \\ \hline
       %  9.3 & 21.03,016 & nA\\ \hline 
       % 9.3.1 & 31.03.2016 & nA\\ \hline
       % 9.3.2 & 17.05.2016 & nA \\ \hline
        \end{tabular} 
        \caption{Auflistung iOS Version/ Veröffentlichungsdatum/ \# Tage bis zum JB}
        \label{tab:iOSVersion}
    \end{center}
\end{table}

Die Tabelle \ref{tab:iOSVersion} listet die iOS Version, das Veröffentlichungsdatum dieser iOS Version und die Anzahl an Tagen die benötigt wurden, um einen JB für diese iOS Version zu veröffentlichen auf.   
\begin{figure}[htbp]
        \centering
                \includegraphics[scale=0.55]{Bilder/iDeviceJB-SW-HW.png}
         \caption{Vergleich der Anzahl der Tage eines JB für die Prozessoren und für der initialen iOS Version}
        \label{fig:VergleichJBProzessorSW}      
\end{figure}

Anhand der Abbildung \ref{fig:VergleichJBProzessorSW} kann gezeigt werden, dass alle Apple Prozessorarchitekturen einschliesslich des Apple A6 ein Sicherheitsgewinn für die iOS Produkte bedeutet haben. Dieser Schluss ist darauf begründet, dass die JBs für die selbe iOS Version auf älteren Apple Prozessorarchitekturen innerhalb von wenigen Tagen verfügbar war. \par 
Eine Veränderung ist in den Daten ab dem Apple A7 Prozessors sichtbar. Da zwischen dem veröffentlichen des JBs für die iOS Hardware und die iOS Version, nur mehr eine marginale Unterschied vorliegt.
\paragraph{Ergebnisse:} Dies lässt den Schluss zu, dass die Sicherheit und die Vertrauenswürdigkeit der iOS Produkte zum heutigen Zeitpunkt nur mehr vom mobilen Betriebssystem von Apple abhängt und nicht von der verwendeten iOS Hardware. 

\newpage
%------------------------------------------------------------------------------
%------------------------------ 
\subsection{iOS Version}
\label{sec:Frage1iOSVersion} 

\begin{figure}[hp!]
        \centering
                \includegraphics[scale=0.35]{Bilder/Frage1_1.png}
        \caption{iOS Version / JB Tools /\# Tage bis zum JB}
        \label{fig:AnalyseiOSJB1}        
\end{figure}

\begin{figure}[hp!]
        \centering
                \includegraphics[scale=0.35]{Bilder/Frage1_2.png}
        \caption{iOS Version / JB Tools /\# Tage bis zum JB}
        \label{fig:AnalyseiOSJB2}
\end{figure}

Die Abbildungen \ref{fig:AnalyseiOSJB1} und \ref{fig:AnalyseiOSJB2} zeigen die bekanntesten und stabilsten untethered JBs, in Abhängigkeit mit der iOS Version und der Anzahl an Tagen die benötigt wurden, um für diese iOS Version ein Jailbreak bereitzustellen. \par 
\textbf{Die Graphiken zeigen,} dass nach einem gelungen JB die Sicherheitsupdates von Apple einen kurzzeitigen Sicherheitsmehrwert mit sich bringen, aber innerhalb einiger Tage wurde ein neuer JB, des selben JB-Team veröffentlicht. Dies lässt darauf schliessen, dass die Sicherheitslücken nur teilweise geschlossen worden sind. Dieser Verhalten gilt  für alle iOS Versionen bis einschliesslich der iOS Version 9.x.\par  
Ab der iOS Version 8.x ändert sich dieses Verhalten, da die JB Community jetzt länger für die JBs der nachfolgenden iOS minor Releases benötigt. \par 
Markant ändern sich die Muster ab der iOS Version 9.x. Das JB für die Minor iOS Version 9.1 benötigte fast sechsmal länger, als das JB für die Major iOS Version 9.0 benötigte.\par



\begin{figure}[hp!]
        \centering
                \includegraphics[scale=0.5]{Bilder/iOSJB1.png}
        \caption{Anzahl der veröffentlichten JBs pro iOS Version 3.x - 5.1.1}
        \label{fig:AnalyseAnzahliOSJB1}
\end{figure}

\textbf{Die Abbildungen \ref{fig:AnalyseAnzahliOSJB1} und \ref{fig:AnalyseAnzahliOSJB2} zeigen,} dass die Anzahl der veröffentlichten JB-Tools, ab der iOS Version 7.0 massiv rückgängig sind. Es wurden nur mehr ein oder zwei JB-Tools pro iOS Version der Öffentlichkeit zur Verfügung gestellt. 

\begin{figure}[hp!]
        \centering
                \includegraphics[scale=0.5]{Bilder/iOSJB2.png}
        \caption{Anzahl der veröffentlichten JBs pro iOS Version 6.x - 9.x.x}
        \label{fig:AnalyseAnzahliOSJB2}
\end{figure}


\paragraph{Ergebnisse:} Die aufbereitet Daten lassen den Schluss zu, dass die Sicherheit und die Vertrauenswürdigkeit der iOS Produkte mit steigender iOS Version zunimmt. \par 
Es muss darauf hingewissen werden, dass Apple seine Strategie im Bezug auf die JB Community geändert hat. In der Vergangenheit hat Apple versucht, gegen JBs gerichtlich vorzugehen, aber nur mit begrenzten Erfolg (\textit{\glqq Klage gegen George Hotz alias Geohot\grqq{}}). Apple hat vor einigen Jahren damit begonnen namhafte iOS Hacker anzustellen, unteranderem Comex  der Entwickler von \textit{\glqq JailbreakMe\grqq}. Für Apple hat dies zwei Vorteile, ersten die Hacker kennen das mobile Betriebssystem und die Schwächen dieses, besser als viel interne Apple Entwickler.  Zweitens, die JB Community verliert ein aktives Mitglied. Als sehr viel schwerwiegender einzustufen ist, dass durch den ehemaligen Hacker auch eine Anzahl an Zero-Day Bugs Apple gemeldet werden.\par
Seit dem Jahr 2015 ist der kommerzielle Aspekt in diesem Zusammenhang nicht ausser acht zu lassen. Angefangen mit dem Preisgeld auf ein JB der iOS Version 8.x bis hin zu Premieren für Zero-Day Bugs. Dies hat meiner Meinung nach einen grossen Einfluss auf die Dauer der Veröffentlichung des JBs. Die markanten Ausreißer in den Metriken, ab der iOS Version 8.x beruhen meiner Meinung nach auf dem kommerziellen Faktor. Der letzte JB wurde Anfang Dezember 2015 veröffentlicht, seit diesem Zeitpunkt wurden sechs iOS Versionen veröffentlicht und es sind 243 Tage seit es letzten JBs verstrichen.


\newpage
%------------------------------------------------------------------------------
%------------------------------ 
\section{Welche Auswirkung haben die von Apple eingeführten Sicherheitsmechanismen und Sicherheitsupdates auf die Sicherheit des Systems?}
\label{sec:Frage2}
%------------------------------------------------------------------------------
%------------------------------ 
\subsection{iOS Sicherheitsmechanismen}
\label{sec:Frage2SecMechanismen}
 
\begin{table}[htp!]
    \begin{center}
        \begin{tabular}{|l|l|l|l|} \hline
            \textbf{Sicherheitsmechanismus} & \textbf{iOS 2.0} & \textbf{iOS 4.3} & \textbf{iOS 8.0} \\ \hline
             Anzahl behobener Fehler & nA & 12 & 48\\ \hline
             MAC & 8 & - & - \\ \hline
             JIT & 8 & 85 & - \\ \hline
             ASLR & - & 85 & - \\ \hline
             verpflichtende Datenverschlüsselung & - & - & 35 \\ \hline
             verpflichtende Datenverschlüsselung & - & - & 73\\ \hline
        \end{tabular} 
        \caption{Auflistung Sicherheitsmechanismus}
        \label{tab:SecMechanismBugs}
    \end{center}
\end{table}

%------------------------------------------------------------------------------
%------------------------------ 
\newpage
\subsection{iOS Sicherheitsupdates}
\label{sec:Frage2SecUpdate}

In diesem Kapitel wird versucht, die Wechselbeziehung zwischen der Sicherheit des iOS Device und den JBs, in Abhängigkeit der Hypothese H1, darzustellen. Im Detail wird die Historie einzelner JB Tool betrachtet. 

\begin{table}[hp!]
    \begin{center}
        \begin{tabular}{| p{20mm} | p{12mm} | p{17mm} | p{25mm} | p{32mm} | p{22mm} | p{15mm} |} \hline
             \textbf{Datum iOS} & \textbf{iOS} & \textbf{\# Bugs} & \textbf{\# JB Bugs} & \textbf{JB Tool} & \textbf{JB Datum} & \textbf{\# Tage bis JB} \\ \hline 
            10.02.2011 & 4.2.6 &  - & -  & JailbreakMe 3.0 & 02.06.2011 & 112 \\ \hline
             09.03.2011 & 4.3 & 12 & 0 & JailbreakMe 3.0 &	02.06.2011 & 85 \\ \hline
             25.03.2011 & 4.3.1 &  - & - & JailbreakMe 3.0 & 02.06.2011 & 69 \\ \hline
            14.04.2011 & 4.2.7 &  4 & 0 & JailbreakMe 3.0 & 02.06.2011 & 49 \\ \hline
             15.04.2011 & 4.3.2 & 5 & 0 & JailbreakMe 3.0 & 02.06.2011 & 48 \\ \hline
             04.05.2011 & 4.2.8 &  - & - & JailbreakMe 3.0 & 02.06.2011 & 29 \\ \hline
            \textbf{04.05.2011} & \textbf{4.3.3} &  - & -  & \textbf{JailbreakMe 3.0} & \textbf{02.06.2011} & \textbf{29} \\ \hline
            15.07.2011 & 4.3.4 &  1 & 2	 & - & - & - \\ \hline
        \end{tabular} 
        \caption{Analyse JB-Tool JailbreakMe 3.0 \protect\footnotemark}         
        \label{tab:AnalyseJailbreakMe3.0}
    \end{center}
\end{table}
%iOS2011-2012
\footnotetext{\url{https://support.apple.com/de-de/HT204611}}
%\footnote{\label{foot:iOS2011-2012}{\url{https://support.apple.com/de-de/HT204611}}}

\paragraph{JailbreakMe 3.0} ist einen \textit{\glqq webbasierter Userland Exploit\grqq{}} und wurde am 02.06.2011 veröffentlicht. Dieser untethered Jailbreak wurde für die iOS Version 4.3.3 bereitgestellt. Es konnten aber mit diesem Exploit die iOS Version 4.2.6-4.3.3 (Siehe Tabelle: \ref{tab:AnalyseJailbreakMe3.0})  \textit{\glqq gejailbreaked\grqq{}} werden. 
Dieser Exploit nutzte einen Fehler im CoreGraphik Framework aus. Dieser ermöglicht es, im Zusammenhang mit dem Lesen eines PDFs, einen beliebigen Code auszuführen. Der IOMobileFrameBuffer hat einen Softwarefehler, welcher es JailbreakMe 3.0 Exploit ermöglichte, Systemprivilegien zu erhalten, Beide Bugs wurde in der iOS Version 4.3.4 \footnote{\label{foot:iOS4.3.4}{\url{https://support.apple.com/de-de/HT202272}}} geschlossen. Ab der iOS Version 4.3.4 ist dieser JB nicht mehr funktionsfähig.
 
\paragraph{Ergebnisse:} Interessant an den Daten der Tabelle \ref{tab:AnalyseJailbreakMe3.0} ist, dass in einem Zeitraum von 112 Tagen sechs iOS Sicherheitsupdates von Apple zur Verfügung gestellt wurden. In dieses sechs Updates wurden insgesamt 24 Bugs behoben. Keiner der behobenen Bugs wurden jemals in einem JB verwendet. Am 15.07.2011 wurde das iOS Sicherheitsupdate 4.3.4 zum Download zur Verfügung gestellt. In diesem Update wurden insgesamt 3 Bugs beschlossen. JailbreakMe 3.0 verwendetet zwei von diesen Bugs und somit war der JB für alle weiteren iOS Version unterbunden. Dies zeigt das Apple die JB Tools \textit{\glqq reverse engineered\grqq{}} und mit dem nächsten Update wurde der JailbreakMe 3.0 JB für alle weiteren iOS Versionen unterbunden. Apple benötigte \textbf{43 Tage} um das Update bereitzustellen.


\newpage
\begin{table}[hp!]
    \begin{center}
        \begin{tabular}{| p{15mm} | p{20mm} | p{17mm} | p{25mm} | p{20mm} | p{22mm} | p{15mm} |} \hline
            \textbf{iOS} & \textbf{Datum iOS} & \textbf{\# Bugs} & \textbf{\# JB Bugs} & \textbf{JB Tool} & \textbf{JB Datum} & \textbf{\# Tage bis JB} \\ \hline 
7.1 & 10.03.2014 & 20 & 4 & Pangu & 23.6.2014 & 105 \\ \hline
\textbf{7.1.1} & \textbf{22.04.2014} & \textbf{4} & \textbf{0} & \textbf{Pangu} & \textbf{23.6.2014} & \textbf{62} \\ \hline
7.1.2 & 30.06.2014 & 18 & 0 & Pangu & 23.6.2014 & -7 \\ \hline
 & & & & & & \\ \hline
8.0 & 17.09.2014 & 44 & 4 & Pangu8 & 22.10.2014 & 35 \\ \hline
8.0.1 & 24.09.2014 & - & - & Pangu8 & 22.10.2014 & 28 \\ \hline
8.0.2 & 25.09.2014 & - & - & Pangu8 & 22.10.2014 & 27 \\ \hline
\textbf{8.1} & \textbf{20.10.2014} & \textbf{5} & \textbf{0} & \textbf{Pangu8} & \textbf{22.10.2014} & \textbf{2} \\ \hline
8.1.1 & 17.11.2014 & 5 & 3 & - & - & - \\ \hline
 & & & & & & \\ \hline
9.0& 16.09.2015 & 71 & 1 & Pangu9 & 14.10.2015 & 28  \\ \hline
\textbf{9.0.1} & \textbf{23.09.2015} & \textbf{-} & \textbf{-} & \textbf{Pangu9} & \textbf{14.10.2015} & \textbf{21}\\ \hline
9.0.2 & 30.09.2015 & 1 & 0 & Pangu9 & 14.10.2015 & -14 \\ \hline
9.1(32 Bit) & 21.10.2015 & 25 & 2 & Pangu9 & 14.10.2015 & -  \\ \hline
		 & & & & & & \\ \hline				
\textbf{9.1(64 Bit)} & \textbf{21.10.2015} & \textbf{25} & \textbf{2} & \textbf{Pangu9} & \textbf{11.3.2016} & \textbf{142}  \\ \hline
9.2 & 08.12.2015	 & 27 & 3 & - & - & - \\ \hline		
     \end{tabular} 
        \caption{Analyse JB-Tool Pangu \protect\footnotemark}
        \label{tab:AnalysePangu}
    \end{center}
\end{table}
%{\label{foot:iOS2015-2016}
%\footnote{\url{https://support.apple.com/de-de/HT201222}}
\footnotetext{\url{https://support.apple.com/de-de/HT201222}}

\paragraph{Das JB Team Pangu} veröffentlicht in den letzten zwei Jahren für drei Major iOS Versionen JBs. Bevor das Team Pangu ein JB veröffentlichten, nahmen die Member des Teams an einem iOS Exploitation Trainings von von Stefan Essers teil \footnote{\url{https://www.sektioneins.de/en/trainings/iosexploitation.html}}. \par
Bis zum heutigen Datum (18.07.2016) wurden vom Pangu Team vier unterschiedliche JB Version veröffentlicht. Mit diesen vier JBs können zwölf iOS Versionen \textit{\glqq gejailbreaked\grqq{}} werden (Siehe Tabelle: \ref{tab:AnalysePangu}).
Die Tabelle \ref{tab:AnalysePangu} zeigt die selbe Wechselbeziehung zwischen den iOS Updates und den JBs,, wie beim JB JailbreakMe 3.0. 
Es ist noch anzuführen, dass das Pangu Team das erste JB für die 64 Bit Architektur veröffentlicht. 
\paragraph{Ergebnisse:}  Die iOS Version 7.1.2 ist das letzte Update für das iPhone 4. Für dieses Device, ist somit das letzte verfügbare Update, als nicht sicher einzustufen. Interessant ist auch, dass das Sicherheitsupdate 8.1.1 drei Softwarefehler behebt die von einem JB verwendet wurden. Alle drei Bugs wurden vom Pangu8 JB ausgenutzt und bedeuteten das Ende des Pangu8 JBs. 

Nach \textbf{26 Tage} brachte Apple ein Sicherheitsupdate heraus und das JB Pangu8 konnte die iOS Version 8.1.1 nicht mehr jailbreaken.Die iOS Version 9.2 beendete nach \textbf{55 Tagen} die Funktionsfähigkeit des Pangu9 JBs.



\newpage
\begin{table}[htp!]
    \begin{center}
        \begin{tabular}{| p{10mm} | p{22mm} | p{17mm} | p{25mm} | p{18mm} | p{22mm} | p{15mm} |} \hline
            \textbf{iOS} & \textbf{Datum iOS} & \textbf{\# Bugs} & \textbf{\# JB Bugs} & \textbf{JB Tool} & \textbf{JB Datum} & \textbf{\# Tage bis JB} \\ \hline 
8.0 & 17.09.2014 & 44 & 4 & TaiG & 29.11.2014 & 73  \\ \hline
8.0.1 & 24.09.2014	& - & - & TaiG & 29.11.2014 & 66 \\ \hline
8.0.2 & 25.09.2014 & - & -  & TaiG & 29.11.2014 & 65  \\ \hline
8.1 & 20.10.2014 & 5 & 0 & TaiG & 29.11.2014 & 40  \\ \hline
\textbf{8.1.1} & \textbf{17.11.2014} & \textbf{5} & \textbf{3} & \textbf{TaiG} & \textbf{29.11.2014} & \textbf{12}  \\ \hline
 & & & & & & \\ \hline						
\textbf{8.1.2} & \textbf{09.12.2014} & \textbf{1} & \textbf{0} & \textbf{TaiG v2} & \textbf{08.03.2015} & \textbf{89}  \\ \hline
	 & & & & & & \\ \hline						
8.1.3 & 27.01.2015 & 16 & 5 & TaiG v3 & 23.06.2015 & 147  \\ \hline
8.2  & 09.03.2015 & 5 & 1 & TaiG v3 & 23.06.2015 & 106 \\ \hline
\textbf{8.3} &  \textbf{08.04.2015} & \textbf{40} & \textbf{1} & \textbf{TaiG v3} & \textbf{23.06.2015} & \textbf{103}  \\ \hline
		 & & & & & & \\ \hline					
\textbf{8.4} &  \textbf{30.06.2015} & \textbf{25} & \textbf{0} & \textbf{TaiG v4} & \textbf{01.07.2015} & \textbf{1}  \\ \hline
8.4.1 & 13.08.2015 & 35 & 9 & - & - & -   \\ \hline
        \end{tabular} 
        \caption{Analyse JB-Tool TaiG \protect\footnotemark}
        \label{tab:AnalyseTaig}
    \end{center}
\end{table}
%{\label{foot:iOS2014}
%\footnote{\url{https://support.apple.com/de-de/HT205762}
\footnotetext{\url{https://support.apple.com/de-de/HT205762}}

\paragraph{TaiG} zeigt wie kein anderes JB-Tool wie sehr Apple und JB-Community Katz und Maus spielen. Apple antwortet innerhalb weniger Tage mit einem Update, welches das JB schliesst. Das JB Team TaiG veröffentlicht für jedes Update ein neues JB. Bis am Ende Apple das iOS Sicherheitsupdate 8.4.1 veröffentlicht. In diesem Update werden 8 Bugs geschlossen, auf denen der TaiG JB aufbaute. 
 

Taig 10 Tage
Taig v2 war schon bei der Veröffentlichung nicht mehr obsulet. 
Taig v3 7 Tage
Die iOS Version 8.4.1 beendete nach \textbf{43 Tagen} die Funktionsfähigkeit des JB Taigv4.
\newpage
%------------------------------------------------------------------------------
%------------------------------ 
\section{Analyse der Hypothese H1}
\label{sec:AnalyseHypo}




        